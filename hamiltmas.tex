\documentclass[11pt,a4paper,oneside,english]{article}
\usepackage{latexstyles/print}

\title{
	Deliberative Subjectivities on Economic Integration in the European Union\\*
	---\\*
	A Draft Project Proposal for the Europa-Universität Flensburg
	}
\author{
	\href{http://www.maxheld.de}{Maximilian Held}
}
\date{
	\today
}

\addbibresource{../held-library/held_library.bib}
%\graphicspath{/img/}

\begin{document}

\maketitle

\begin{abstract}
	\immediate\write18{pandoc README.md -t latex -o Output/tmp.tex}%
	\input{tmp.tex}
\end{abstract}

\newpage

\epigraph{
	``The Secretary, after mature reflection on this point, entertains a full conviction, that an assumption of the debts of the particular states by the union, and a like provision for them, as for those of the union, will be a measure of sound policy and substantial justice''\\*
	--- Alexander Hamilton, first US treasury secretary, 14 January 1790
}

\epigraph{
    ``The unforced force of the better argument:
    [\ldots] \\*
    %\cite[305]{Habermas1996}
    The speaker must choose a comprehensible expression so that speaker and hearer can understand one another.
    [\ldots] \\*
 	%\cite[2f]{Habermas1976}
	Anyone acting communicatively must, in performing any speech act, raise universal validity claims and suppose that they can be vindicated.''\\*
	% % %\cite[2]{Habermas1979}
    --- Jürgen Habermas (\citeyear[305]{Habermas1996}, \citeyear[2f]{Habermas1976}, \citeyear[2]{Habermas1979}, respectively)
}

\newpage

\tableofcontents

\newpage

\section[Background]{Background} \label{sec:background}

\begin{landscape}
 \begin{figure}[htbp]
    \begin{center}
	\includegraphics[width=1\linewidth]{img/deliberative-subjectivities}
	\caption{Conceptual Map of the Research Project ``Deliberative Subjectivities on Economic Integration (In the EU)''}
	\label{fig:deliberative-subjectivities}
	\end{center}
\end{figure}
\end{landscape}

\subsection[European Integration]{European Integration} \label{sec:european-integration}

\paragraph{The EU-25 as a defunct mixed economy.}

\paragraph{The End Game}

\subsection[Deliberative Experiments]{Experiments Deliberative Democracy} \label{sec:deliberative-experiments}

\paragraph{Q Methodology}

\subsection[Philosophy of Economics]{Philosophy of Economics} \label{sec:philosophy-economics}

\section[Research Questions]{Research Questions} \label{sec:research-questions}

\section[Project]{Project} \label{sec:project}

\begin{landscape}
 \begin{figure}[htbp]
    \begin{center}
	\includegraphics[width=1\linewidth]{img/euf-partners}
	\caption{Possible Cooperation Partners for Research Project}
	\label{fig:euf-partners}
	\end{center}
\end{figure}
\end{landscape}

\subsection[Teaching]{Teaching} \label{sec:teaching}

\subsection[Outreach]{Outreach \& Civics Education} \label{sec:outreach}

\subsection[Research]{Research} \label{sec:research}

\section[Job Profile]{Job Profile} \label{sec:job-profile}




%tax competition (PD)?

%include Caplan, that's the guy
%cite the deliberative poll on EU!

%concourse sampling strategy for eu, too? (table)



% Ökonomische Schlussfolgerungen:
% \begin{enumerate}
% 	\item Effiziente und faire wirtschaftliche Integration braucht immer eine \alert{intakte Mischökonomie mit unionsweiten Steuern}.
% 	\item Das Wohlstands- und Produktivitätsgefälle in der EU macht einheitliche Steuersätze einstweilen unmöglich; wir brauchen eine \alert{Transferunion}.
% 	\item Reales Entsparen, Kredit oder Asset-Blasen und Inflation können Krisen verstecken, verschieben und verschlimmern.
% 	\alert{Nichts ist gut, weil nominelle Variablen wie BSP oder Beschäftigung gut aussehen.}
% 	\item Kapitalmärkte mögen bessere Regulierung brauchen; \alert{Finanzkrisen sind aber letztlich Epiphänomene}.
% \end{enumerate}

%note the preliminary review-type work that already exists
%Reference eurolektionen work
%note that a similar effort should also be undertaken for the financial crisis and systemic *risk*; same deal there.
	%maybe I even ought to do that earlier.
%highlights in green
%include dimensions of human need, failures table

%twin crises

%and dual crises links to political system and vice versa

%What kind of an economic reality results from this open, but heterogeneous EU, with unbounded trade, mis-configured currency union and rampant tax competition?
%BoP logic is important, so is Haig-Simons identity
%aaaand comparison to ideal mixed economy

%not all of these ideas for collaboration will work out; local knowledge always matters, and collaboration is best discussed over a coffe table, not planned at a reissbrett.

%deeper disagreement might play out on economic nationalism ("identity")
	%as well as the definition of "pragmatism" itself, i.e. which stuff you can ask for.
	%In pension design, as elsewhere in public policy, a Haig-Simons understanding of the economy helps us to sift through the epiphenomenal debates (“funded” vs. PAYGO), to relegate the complex details (financial markets) to appropriate theory and data, and advance to those choices that our scarce, constrained and material world leaves us to take: how much we should save for future gen- erations and in which form, and who of us, rich or poor, should contribute how much. T

% This is Pangloss at his finest: if you assume, as modern-day second- besters do, that the very means to deliberately get to a better world — government and democracy — are inevitably flawed, you can show, with almost hermetic logic, that whatever world we find ourselves in, must be the best of all possible worlds.
% In that word — possible — lies the catch. Modern-day second-besters assume that, akin to markets and evolution, government and democracy are aimless processes that merely aggregate pre-social, more- or less rational self-interest. If government and democracy are aimless, it follows — as it does, in fact, follow for markets and evolution — that any positive results of government and democracy are beyond reproach, and beyond improvement.

%O↵e might, asked about the plight of German cleaners, point to a leak in the “Keynesian” roof of full employment, protecting the lower storeys of welfare states. Today, full employment is only a neces- sary, but not a su

%Crucially, we must, again, understand that economic integration must always beget more economic and political integration, that, production at great economic scale implies solidarity at the same scale.

%TTIP!

%Democracy needs not just a legitimacy of inputs, but also of outputs (on Europe see Scharpf 1999). In Dahl’s (1994) terms, democracies need to be “system e↵ective”, in Zu ̈rn’s (2000) apt words, democracies need to be “output congruent”: people must be able to choose any (liberal) policy they want to govern a given polity. The EU, currently violates output congruency: with an impotent, largely defunct mixed economy, the people of Europe cannot have all materially possible policies to improve their currently grim life chances, including crushing trade imbalances, demoralizing structural unemployment, wild economic cycles, public squalor and rampant inequality. In the half-built, supposedly sui generis glass of Europe, there is a wide mismatch between the two walls: the exchange components of the mixed economy roam the continent, while most of the command components are confined to the constrained nation state. As a result, the glass is heavily leaking water, both in e

%[...] that every interim solutions between the extremes of intact national sovereignty on the hand, and complete european supra- nationalty of a European Federation will, inevitably, violate both the reference point of welfare state protection and that of demo- cratic legitimacy.227 — O↵e (1998: 41)

%It does not take another genius to, as Keynes (1936) did in 1918, an- ticipate the economic consequences of this particular peace, and to recog- nize how fragile this mode of one-sided, one-legged, market-only European integration is. It is both the greatest strength and greatest weakness that democracy, to thrive and to persist, must be able to complement market pro- duction and distribution with a plan.

%It also seems a little bit unfair to blame it all on Europe. Negative is the current mode of worldwide economic integration, not just in Europe. The EU and its MS play the same PD games not just on this continent, but on higher levels with other countries. The contradictions of a liberalized world economy without a world government are not merely european problems.

%Genschel, too, reminds us of what Pangloss would rather have us forget: “The e↵ect [of globalization] is not so much to force change upon the tax [and thereby, welfare] state as to reduce its freedom to change.” (2005: 53).

%explain the alternative first

%vnM rationality link between ordinal preferences and cardinal utility.

%ideal observation Rawls 1998

%why go this deep on european questions? because EU integration is the economic modernization question of our age, as maybe, 200 years ago, it was the dissolution of status priviliges.

%Pragmatism

%table comparing deliberative and aggregative democracy

%there is not much else beside a priori knowledge we might learn.

%copy from the "difference" section of europe

%# Habermas Ach Europa
% so clearly, we want only a notion of a plausible *possibility* of rational solutions, we don't want to look just at rational solutions (first vs second order).
% However, you can't just defer this to process; that would be insufficient, and turn full circle. A deliberation is good if it makes rational solutions plausible, and solutions are plausible if they result from good deliberation. Discuss the different operationalizations about this, including DQI.
% So what we need, is some kind of standard that is not procedural, but also not substantive, but in-between, kinda like  Rawls. But we don't have that ideal position (but we might approximate it?)

% The question now is: can Q-method, or intersubjective rationality and meta-consensus *be* this thing, and if so, why.

% Need to look in to the whole concourse theory thing

% Q Methodology, maybe via its quantum theoretic reading (it's about viewpoints, and there exist no objective statements (like zero energy atom states) outside of it -- that's not soo far from discourse theory.
% 	on this see especially Wendy S. Rogers in operating subjectivity; she argues it is and/or can be used as a form of discourse analysis.

% FIXME I might have the difference between constructivism and constructionism wrong in my thesis, especially when citing Berger/Luckmann which seem to be the former and Foucault which is the latter. Constructionism may be the social variant.

% Write to McCafferys co-author about q, since he is a psychologist and also does quant stuff.

% Somehow the stuff that happens during deliberation is actually quite close to some of the stuff that q methodologists rights and that stephenson assumed; it's in expressing and exchanging views vis-a-vis subjects. Both approaches negate that there's stuff "in the braiN" that just need be extracted.

%table assumptions perfect competition

%second and first order questions

%stress the feat of cooperation, or positive sum gains, literature too; that (in the form of positive economies of scale) is by definition our way out of the Malthusian crises; "this achievement can hardly be overstated"
%the broad question of economic integration that lurks here can hardly be overstated; and not only in the familiar "peace after Weimar" rhetoric; though that works, too.
%enumerate some (preliminary) axiomatic problems with economics:
	%comparative statics is only one, simplistic way to think about it
	%first theorem of welfare economics starts from *given* distributions; that's crucially not the case in Europe
	%
%note the emptyness of economic welfare logic – but what can be put in its place?

% \begin{enumerate}
% 	\item Die Nutzen, Kosten und Bedingungen der wirtschaftlichen Integration müssen \alert{umfassend erklärt} werden.
% 	Zur wirtschaftlichen Integration und dem \alert{Abschied von imaginierten Gemeinschaften} wie dem Nationalstaat gibt es keine attraktive Alternative.
% 	\item Vollbeschäftigung (links, Nachfrageseite) \emph{und} BSP (rechts, Angebotseite) \alert{taugen beide nicht als \emph{hinreichende Ziele} für gute Politik}.
% 	\item \emph{Vielleicht} geht bei wirtschaftlicher Integration \alert{Tiefe vor Breite}.
% \end{enumerate}

%negative integration sucks

%assumptions-failures list; look at this also in terms of the disagreement

%problems solutions mixed economy -- does that make sense?

%History of ideas suggests ideas are explanans, what *ought to be explained*; but that's not me, I am somewhere in between and the other position (would that be econ?)

%story about the missing leg, generally
%story of redistribution behind economic integration
%really look at FES; this isn't a european problem, but Europe has always been the project to fix this.

%european economic order is missing essential institutions of postwar mixed economies; economic integration in the current mode risks the accomplishments of western welfare state, disturbs the historic balance between efficiency and equity goals and creates macroeconomic imbalances which unload in periodic crises.

%fixes
	%angleichung von faktorpreisen
	%agglomeration and NTT
	%

%On this one leg, rergional integration has occured folloing the rolemodel of national economies; the common market guarantees factor (labor, capital) and goods mobility (products, services), effectively regulated by the EP and the Commission.

%there is a deeper research project behind this, for which these different economic cases are, well cases.

%Ziel der Europa-Universität Flensburg ist es, den Prozess der europäischen Integration in Forschung, Lehre und Weiterbildung zu begleiten und planvoll zu befördern. Die Universität verfolgt den Anspruch, in ihren Studiengängen eine Generation auszubilden, die Europa versteht, lebt und die Fortentwicklung Europas in Theorie und Praxis vorantreibt. [website]

%Die Forschung an der Europa-Universität Flensburg konzentriert sich auf die Schwerpunkte Bildung, Wirtschaftswissenschaften, Umweltwissenschaften mit einem Fokus auf den interdisziplinären Bereich Nachhaltige Entwicklung und interdisziplinäre Europastudien. [website]
	%bingo; genau in der Schnittmenge liegt das Projekt

%Zweimal unterrichten; 1x als DSA-Kurs, 1x an schulen, 1x als CiviCon?

%interdisziplinäre Europastudien

%mit lehrerbildung kooperieren. – als pilotprojekt, und um die didaktische Konzeption zu schärfen
	%EULE - lehrerbildung
	%welche fachbereiche in der lehrerbildung gibt es?
	%pilotversuche an Partnerschulen
	%im unterschied zu gegenwärtiger Europaschulung geht es *nicht* um institutionkunde, sondern um die (ökonomischen) Abstraktionen der Einigung
	%Vergleich auch Eurolektionen!
	%Drittmittel evtl. aus Horizon 2020
	%Bürger-krams:
		%http://hhsh.enterprise-europe-germany.de/public/uploads/een-hhsh/downloads/KurzinfoEuropafuerBuergerInnen.pdf
	% some kinds of venn diagrams with european integration, mixed economy, philosophy of economics, deliberative subjectivities of the economy in the center
	% also diagram with inputs, outputs etc.
	%also diagram: (maybe downward thing, links ziel / Rechts geldgeber etc.
		%1. FORSCHUNG Europa als (dys)funktionale Mischwirtschaft
		%2. LEHRE forschendes lernen / gemeinsames schreiben
			%Studiengang MA European Studies
			%Studiengang MA Intl Management
			%Studiengang FH: wirtschaftsinfo?
			%Studiengang BA Bildungswissenschaften / MA Ed Gemeinschaftsschule\tabularnewline
			%Gesellschaftswissenschaften, ABER AUCH AESTHETISCHE BILDUNG!
			%vorhab github-schulung
			%verweis auf forschendes lernen der studierenden (lehrer) von der TITTF.
			%SEMINAR: schreiben auf GIT/LATEX?!
			%SEMINAR: q-methode und andere empirische sozialforschung / umfragen
			%SEMINAR: Mischwirtschaft etc. (2 Semester, ggfs, mit Mikro- Makro einführung erforderlich)
		%3. PROJEKT an lokalen Schulen (eurolektionen)
		%4. LEHRERBILDUNG
			%fachdidaktik Wirtschaft, Politik, Europa
		%5. CIVICON
		%6. FORSCHUNG: Q-Studien abgreifen
		%auch aufschreiben: Planspiele, weitere didaktische Übungen, sowie TANZEN etc.
%Institut für Ästhetisch-Kulturelle Bildung
%Institut für Erziehungswissenschaften:
	%darin vor allem Schulbildung
%Institut für Gesellschaftwissenschaften und Theologie
	%Seminar für Politikwissenschaft und Politikdidaktik
	%Seminar für Geschichte und Geschichtsdidaktik
%Internationales Institut für Management und ökonomische Bildung
	%Abteilung Wirtschaftswissenschaften und ihre Didaktik
	%Abteilung Sozial- und Bildungsökonomie
	%Abteilung Internationale und Institutionelle Ökonomik
	%Abteilung Finanzwirtschaft
	%Abteilung Europa- und Völkerrecht
%Interdisziplinäres Institut für Umwelt-, Sozial- und Humanwissenschaften
	%darin vor allem Abteilung Zentrale Methodenlehre
%Kooperationsschulen an der Uni Flensburg
%Zentrum für Lehrerbildung?
%vielleicht als interdisziplinäres Projekt im Rahmen des Master of Education (Gemeinschaftsschule)

%grözinger: mitbestimmung bei der ECB?

% Grundsätzliche alternativen im verstehen der ökonomie; etwa Böll, Institutionenökonomie etc. Etwa eine "europäische dingsda politik" oder die idee des wettbewerbes lassen (im besten Fall) sich zurückführen auf grundsetzlichen dissens, oder totales missverständnis von "unkonzepten", oder eben auf interessierte Ideen (idea research) – but you can't always assume that.


% Gerd Grözinger
	%conflict between economic efficiency and political legitimacy remains (this is the booktitle)

%ausschreibung: Bereich schule/unterricht sowie europäische/euro-wiki relevanz
%Projektskizze
%sowie: profil der auszuschreibenden Stelle

% ---

% Title: Deliberative Subjectivities of Economic Integration

% Background
% The case (but also more than a case): European Integration
% Compared against an idealised mixed economy, the European Union (EU) displays a sharp imbalance between deeply integrated markets (the common market), but embryonic transnational economic governance. The absence of tax harmonisation, and a union-wide management of aggregate demand (AD) in particular cause unintended market-state interactions. Much of the current sovereign debt and banking crises as well as the resulting economic slump can be understood as the cyclical outgrows of these structural dysfunctions (in addition to microprodutential dysfunctions in financial markets).
% (this diagnosis, admittedly, starts from neoclassical and ordoliberal assumptions; these assumptions themselves need to be problematised; see below meta-theory)
% Behind these crises of an incomplete mixed economy lurk the deeper question of how economic convergence in a starkly unequal union is to be organised, and politically legitimated.
% The political debate (austerity vs profligacy, moral hazard) appears to bear little resemblance to some of the economic abstractions (perfect currency unions, systems competition, etc.) central to understanding these crises.
% The method (but also more than a method): Deliberative Democracy
% The empirical study of deliberative democracy investigates if, and how, people can reach different (more rational?) decisions, if they participate in prolonged, egalitarian, common-good oriented, mutually-understanding discussion.
% Cases: Existing research focuses mostly on local, non-abstract issues (say, waste disposal location), and short formats (half a day).
% Methods: Methodological approaches to studying deliberative democracy fall broadly into two camps; 1) quantitative, survey-type instruments and 2) qualitative, discourse-analytic approaches. Both fall short of approaching a substantive (not merely procedural) standard of deliberation, as espoused by Habermas. Quantitative approaches reduce deliberation to knowledge gain and attitude change (Fishkin), and (critical) discourse analyses put researchers in the awkward position of adjudicating substantive disagreements.
% Niemeyer and Drysek have suggested an attractive, alternative operationalisation of deliberation: good deliberation has happened, when people prefer same policies for the same value- and belief-reasons (and vice versa) (intersubjective rationality), and agree on the realm of alternative preferences, values and beliefs (metaconsensus). This can be measured with Q-Methodology, a (sort-of) inverted factor analysis of sets of around 60-90 statements that participants are asked to rank-order before and after the deliberation, in an experimental setup.
% The (meta-?)theory: Philosophy of economics
% Some apparent popular disagreement on economic policy (in EU integration and elsewhere) may be traced to misunderstandings, and lack of knowledge. There is some empirical research on these systematic misunderstandings (McCaffery, Baron), but - as outgrows of prospect theory (Kahnemann/Tversky) - they tend not to cover high-level, abstracted choices.
% However, a substantial core of economic axioms will remain, the validity of which experts cannot adjudicate themselves (i.e.. to what extent is / ought to be / can pragmatically assumed to be / homo sapiens a homo oec.? What is "added value"? etc.). We need to establish plausible links between these low-level axioms and disagreement on policy choices.
% Moreover, these axioms ought to be deconstructed, and then reconstructed by a discerning body of citizens.
% Research question(s) (in increasing levels of abstraction)
% How would ordinary citizens think differently about european integration, if they had the time to thoroughly inform themselves and deliberate amongst one another?
% Can deliberation work on another (like tax) highly complex issue, with (ideally) an international group of participants?
% What is the structure (=rotated factors) of citizen subjectivities on european integration, and how does this structure change as a result of deliberation?
% Building on my previous work on the subjectivities on tax, what are the broader structures of deliberative subjectivities on economics (meta-factors, or factors at the levels of economic axioms?
% Project components / Links to EUF
% Teaching
% (2-semester) literature seminars (in MA European Studies / MA Intl Management / MA Lehramt Mittelschulen)
% Intensive theory reading; philosophy of economics as applied to the economics of regional integration.
% Collaborative writing (via GitHub / LaTeX) of a handbook for citizens: what are the most important and most controversial abstractions people have to know about regional (EU) integration?
% Pre-Testing of Q-Sorts (students will complete q-sorts before and after their seminar participation)
% (extent/format of this course will depend on prerequisites, ideally Micro-/Macro-Economics, as well as institutional economics – may also be opened for interested BA students)
% (1-semester) methods seminar (in MA European Studies, MA Intl Management, MA Lehramt Mittelschulen)
% review of mainstream empirical social research methods
% how does q-methodology compare?
% development, analysis of q-sample
% (1-semester) didactics seminar (in MA Lehramt Mittelschulen)
% collaboration with teacher education at EUF
% how can the economics of european integration be taught to diverse, international groups following deliberative standards?
% Civil Society Outreach ("Politische Bildung"):
% School project on European integration with local schools, and student teachers (model: Eurolektionen from 2009)
% ideally with danish students, too.
% serves as pre-test with focus on didactics
% cooperation with Lehrerbildung / EULE @ EUF
% limited (internal?) funding required
% Summer school with 16-20 year olds (model: Deutsche SchülerAkademie)
% with german students
% 2 weeks
% serves as pre-test with focus on didactics
% limited (internal?) funding required
% may be scaled up to international summer school, would require substantial funding
% CiviCon on European Integration (model: CiviCon on Taxation)
% A citizen conference on european integration, 2-4 weeks
% ideally with european participants
% would require substantial (external!) funding
% Research
% Publication opportunities from development of handbook, and literature review of philosophy of economics as applied to economic integration in the EU.
% Publication opportunities from iterative administration of q-sorts to students.
% Preparation and acquisition of substantial funding for a CiviCon on European Integration.
% Development of a research project into the deliberative subjectivities on economics.
\printbibliography
\end{document}