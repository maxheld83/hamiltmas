% > The Secretary, after mature reflection on this point, entertains a full conviction, that an assumption of the debts of the particular states by the union, and a like provision for them, as for those of the union, will be a measure of sound policy and substantial justice''
% > -- Alexander Hamilton, first US treasury secretary, 14 January 1790

%History of ideas suggests ideas are explanans, what *ought to be explained*; but that's not me, I am somewhere in between and the other position (would that be econ?)

%story about the missing leg, generally
%story of redistribution behind economic integration
%really look at FES; this isn't a european problem, but Europe has always been the project to fix this.

%european economic order is missing essential institutions of postwar mixed economies; economic integration in the current mode risks the accomplishments of western welfare state, disturbs the historic balance between efficiency and equity goals and creates macroeconomic imbalances which unload in periodic crises.

%fixes
	%angleichung von faktorpreisen
	%agglomeration and NTT
	%

%On this one leg, rergional integration has occured folloing the rolemodel of national economies; the common market guarantees factor (labor, capital) and goods mobility (products, services), effectively regulated by the EP and the Commission.

%there is a deeper research project behind this, for which these different economic cases are, well cases.

%Ziel der Europa-Universität Flensburg ist es, den Prozess der europäischen Integration in Forschung, Lehre und Weiterbildung zu begleiten und planvoll zu befördern. Die Universität verfolgt den Anspruch, in ihren Studiengängen eine Generation auszubilden, die Europa versteht, lebt und die Fortentwicklung Europas in Theorie und Praxis vorantreibt. [website]

%Die Forschung an der Europa-Universität Flensburg konzentriert sich auf die Schwerpunkte Bildung, Wirtschaftswissenschaften, Umweltwissenschaften mit einem Fokus auf den interdisziplinären Bereich Nachhaltige Entwicklung und interdisziplinäre Europastudien. [website]
	%bingo; genau in der Schnittmenge liegt das Projekt

%Zweimal unterrichten; 1x als DSA-Kurs, 1x an schulen, 1x als CiviCon?

%interdisziplinäre Europastudien

%mit lehrerbildung kooperieren. – als pilotprojekt, und um die didaktische Konzeption zu schärfen
	%EULE - lehrerbildung
	%welche fachbereiche in der lehrerbildung gibt es?
	%pilotversuche an Partnerschulen
	%im unterschied zu gegenwärtiger Europaschulung geht es *nicht* um institutionkunde, sondern um die (ökonomischen) Abstraktionen der Einigung
	%Vergleich auch Eurolektionen!
	%Drittmittel evtl. aus Horizon 2020
	%Bürger-krams:
		%http://hhsh.enterprise-europe-germany.de/public/uploads/een-hhsh/downloads/KurzinfoEuropafuerBuergerInnen.pdf
	% some kinds of venn diagrams with european integration, mixed economy, philosophy of economics, deliberative subjectivities of the economy in the center
	% also diagram with inputs, outputs etc.
	%also diagram: (maybe downward thing, links ziel / Rechts geldgeber etc.
		%1. FORSCHUNG Europa als (dys)funktionale Mischwirtschaft
		%2. LEHRE forschendes lernen / gemeinsames schreiben
			%Studiengang MA European Studies
			%Studiengang MA Intl Management
			%Studiengang FH: wirtschaftsinfo?
			%Studiengang BA Bildungswissenschaften / MA Ed Gemeinschaftsschule\tabularnewline
			%Gesellschaftswissenschaften, ABER AUCH AESTHETISCHE BILDUNG!
			%vorhab github-schulung
			%verweis auf forschendes lernen der studierenden (lehrer) von der TITTF.
			%SEMINAR: schreiben auf GIT/LATEX?!
			%SEMINAR: q-methode und andere empirische sozialforschung / umfragen
			%SEMINAR: Mischwirtschaft etc. (2 Semester, ggfs, mit Mikro- Makro einführung erforderlich)
		%3. PROJEKT an lokalen Schulen (eurolektionen)
		%4. LEHRERBILDUNG
			%fachdidaktik Wirtschaft, Politik, Europa
		%5. CIVICON
		%6. FORSCHUNG: Q-Studien abgreifen
		%auch aufschreiben: Planspiele, weitere didaktische Übungen, sowie TANZEN etc.
%Institut für Ästhetisch-Kulturelle Bildung
%Institut für Erziehungswissenschaften:
	%darin vor allem Schulbildung
%Institut für Gesellschaftwissenschaften und Theologie
	%Seminar für Politikwissenschaft und Politikdidaktik
	%Seminar für Geschichte und Geschichtsdidaktik
%Internationales Institut für Management und ökonomische Bildung
	%Abteilung Wirtschaftswissenschaften und ihre Didaktik
	%Abteilung Sozial- und Bildungsökonomie
	%Abteilung Internationale und Institutionelle Ökonomik
	%Abteilung Finanzwirtschaft
	%Abteilung Europa- und Völkerrecht
%Interdisziplinäres Institut für Umwelt-, Sozial- und Humanwissenschaften
	%darin vor allem Abteilung Zentrale Methodenlehre
%Kooperationsschulen an der Uni Flensburg
%Zentrum für Lehrerbildung?
%vielleicht als interdisziplinäres Projekt im Rahmen des Master of Education (Gemeinschaftsschule)

%grözinger: mitbestimmung bei der ECB?

% Grundsätzliche alternativen im verstehen der ökonomie; etwa Böll, Institutionenökonomie etc. Etwa eine "europäische dingsda politik" oder die idee des wettbewerbes lassen (im besten Fall) sich zurückführen auf grundsetzlichen dissens, oder totales missverständnis von "unkonzepten", oder eben auf interessierte Ideen (idea research) – but you can't always assume that.


% Gerd Grözinger
	%conflict between economic efficiency and political legitimacy remains (this is the booktitle)

%ausschreibung: Bereich schule/unterricht sowie europäische/euro-wiki relevanz
%Projektskizze
%sowie: profil der auszuschreibenden Stelle

% ---

% Title: Deliberative Subjectivities of Economic Integration

% Background
% The case (but also more than a case): European Integration
% Compared against an idealised mixed economy, the European Union (EU) displays a sharp imbalance between deeply integrated markets (the common market), but embryonic transnational economic governance. The absence of tax harmonisation, and a union-wide management of aggregate demand (AD) in particular cause unintended market-state interactions. Much of the current sovereign debt and banking crises as well as the resulting economic slump can be understood as the cyclical outgrows of these structural dysfunctions (in addition to microprodutential dysfunctions in financial markets).
% (this diagnosis, admittedly, starts from neoclassical and ordoliberal assumptions; these assumptions themselves need to be problematised; see below meta-theory)
% Behind these crises of an incomplete mixed economy lurk the deeper question of how economic convergence in a starkly unequal union is to be organised, and politically legitimated.
% The political debate (austerity vs profligacy, moral hazard) appears to bear little resemblance to some of the economic abstractions (perfect currency unions, systems competition, etc.) central to understanding these crises.
% The method (but also more than a method): Deliberative Democracy
% The empirical study of deliberative democracy investigates if, and how, people can reach different (more rational?) decisions, if they participate in prolonged, egalitarian, common-good oriented, mutually-understanding discussion.
% Cases: Existing research focuses mostly on local, non-abstract issues (say, waste disposal location), and short formats (half a day).
% Methods: Methodological approaches to studying deliberative democracy fall broadly into two camps; 1) quantitative, survey-type instruments and 2) qualitative, discourse-analytic approaches. Both fall short of approaching a substantive (not merely procedural) standard of deliberation, as espoused by Habermas. Quantitative approaches reduce deliberation to knowledge gain and attitude change (Fishkin), and (critical) discourse analyses put researchers in the awkward position of adjudicating substantive disagreements.
% Niemeyer and Drysek have suggested an attractive, alternative operationalisation of deliberation: good deliberation has happened, when people prefer same policies for the same value- and belief-reasons (and vice versa) (intersubjective rationality), and agree on the realm of alternative preferences, values and beliefs (metaconsensus). This can be measured with Q-Methodology, a (sort-of) inverted factor analysis of sets of around 60-90 statements that participants are asked to rank-order before and after the deliberation, in an experimental setup.
% The (meta-?)theory: Philosophy of economics
% Some apparent popular disagreement on economic policy (in EU integration and elsewhere) may be traced to misunderstandings, and lack of knowledge. There is some empirical research on these systematic misunderstandings (McCaffery, Baron), but - as outgrows of prospect theory (Kahnemann/Tversky) - they tend not to cover high-level, abstracted choices.
% However, a substantial core of economic axioms will remain, the validity of which experts cannot adjudicate themselves (i.e.. to what extent is / ought to be / can pragmatically assumed to be / homo sapiens a homo oec.? What is "added value"? etc.). We need to establish plausible links between these low-level axioms and disagreement on policy choices.
% Moreover, these axioms ought to be deconstructed, and then reconstructed by a discerning body of citizens.
% Research question(s) (in increasing levels of abstraction)
% How would ordinary citizens think differently about european integration, if they had the time to thoroughly inform themselves and deliberate amongst one another?
% Can deliberation work on another (like tax) highly complex issue, with (ideally) an international group of participants?
% What is the structure (=rotated factors) of citizen subjectivities on european integration, and how does this structure change as a result of deliberation?
% Building on my previous work on the subjectivities on tax, what are the broader structures of deliberative subjectivities on economics (meta-factors, or factors at the levels of economic axioms?
% Project components / Links to EUF
% Teaching
% (2-semester) literature seminars (in MA European Studies / MA Intl Management / MA Lehramt Mittelschulen)
% Intensive theory reading; philosophy of economics as applied to the economics of regional integration.
% Collaborative writing (via GitHub / LaTeX) of a handbook for citizens: what are the most important and most controversial abstractions people have to know about regional (EU) integration?
% Pre-Testing of Q-Sorts (students will complete q-sorts before and after their seminar participation)
% (extent/format of this course will depend on prerequisites, ideally Micro-/Macro-Economics, as well as institutional economics – may also be opened for interested BA students)
% (1-semester) methods seminar (in MA European Studies, MA Intl Management, MA Lehramt Mittelschulen)
% review of mainstream empirical social research methods
% how does q-methodology compare?
% development, analysis of q-sample
% (1-semester) didactics seminar (in MA Lehramt Mittelschulen)
% collaboration with teacher education at EUF
% how can the economics of european integration be taught to diverse, international groups following deliberative standards?
% Civil Society Outreach ("Politische Bildung"):
% School project on European integration with local schools, and student teachers (model: Eurolektionen from 2009)
% ideally with danish students, too.
% serves as pre-test with focus on didactics
% cooperation with Lehrerbildung / EULE @ EUF
% limited (internal?) funding required
% Summer school with 16-20 year olds (model: Deutsche SchülerAkademie)
% with german students
% 2 weeks
% serves as pre-test with focus on didactics
% limited (internal?) funding required
% may be scaled up to international summer school, would require substantial funding
% CiviCon on European Integration (model: CiviCon on Taxation)
% A citizen conference on european integration, 2-4 weeks
% ideally with european participants
% would require substantial (external!) funding
% Research
% Publication opportunities from development of handbook, and literature review of philosophy of economics as applied to economic integration in the EU.
% Publication opportunities from iterative administration of q-sorts to students.
% Preparation and acquisition of substantial funding for a CiviCon on European Integration.
% Development of a research project into the deliberative subjectivities on economics.